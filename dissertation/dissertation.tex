\documentclass[12pt, parskip=full, numbers=noenddot]{scrreprt}

% language
\usepackage[english]{babel}
\usepackage[utf8]{inputenc}
\usepackage{csquotes}
\usepackage{url}
\usepackage[version=4]{mhchem}
\usepackage{siunitx}
\DeclareSIUnit\molar{\mole\per\cubic\deci\metre}
\DeclareSIUnit\Molar{\textsc{m}}
\DeclareSIUnit\calorie{cal}

\usepackage{mathptmx} % change font to times

% citations
\usepackage[
  natbib=true,
  backend=biber,
  doi=false,
  isbn=false,
  url=false,
  date=year,
  style=alphabetic,
  citestyle=authoryear]{biblatex}
\addbibresource{dissertation.bib}
\usepackage{varioref}

% floats
\usepackage{graphicx}
  \graphicspath{ {./graphics/} }
\usepackage{subcaption}
\usepackage{tabularx}
  \newcolumntype{L}{>{\raggedright\arraybackslash}X}
\usepackage{booktabs}
\usepackage{pgfplotstable}

\title{Lab meeting}
\author{Arin Wongprommoon}
\date{2019-03-15}

\begin{document}

\maketitle
% Will deal with the intricacies of the cover pages at the very end.

%\tableofcontents

\chapter{EMSA-SELEX}
\label{ch:emsaselex}

\section{Introduction}
\label{sec:emsaselex_intro}

The dinucleotide sequences AA, AT, and TA confer high flexbility to DNA.  Therefore, they are situated on the face of the helical repeat that directly interacts with the histone octamer, with a periodicity of approximately 10 bp \citep{struhl_determinants_2013}.  GC dinucleotides also occur periodically, but out of phase with the aforementioned dinucleotides.  These preferences were confirmed by a yeast-based genome-wide assay, which identified DNA regions stably wrapped in nucleosomes, and the associated dinucleotide probability distributions contributed to a computational model that confirmed these preferences \citep{segal_genomic_2006}.  In contrast, the homopolymeric sequences poly(dA:dT) and poly(dG:dC) confer stiff structures that inhibit nucleosome formation.  These sequences are enriched in linker DNA between nucelosomes.  Additonally, DNA methylation disfavours nucleosome positioning \emph{in vivo}, as it increases the rigidity of DNA \citep{huff_dnmt1-independent_2014}.

To date, clean \emph{in vitro} sequence preference data for CpG methylation is not available.  This is in contrast to sequence preference data derived from MNase-seq on existing genomes in previous studies \citep{struhl_determinants_2013, segal_genomic_2006, huff_dnmt1-independent_2014}.  Furthermore, there has been no exploration into the effects of CH methylation (non-CpG methylation) on sequence preferences of nucleosomes.  Although CpG methylation is the predominant chemical modification of eukaryotic DNA, playing roles in gene regulation and physiology, CH methylation regulates development of stem cells and neurons \citep{guo_distribution_2014}.

SELEX is a promising method to identify the DNA sequences preferred for nucleosome positioning.  Unlike MNase-seq, SELEX employs a random DNA library, which has a higher complexity than genomic sequences and a more homogenous background sequence distribution.  \citet{lowary_new_1998} employed SELEX in uncovering sequence-based rules for nucleosome positioning, but analysis was limited by the technology at the time.  This project extended on their study by employing sequencing to a higher depth, allowing identification of signals with higher information content.

In this project, I aimed to study the sequence specificity of the nucleosome on methylated and non-methylated DNA.  I conducted four cycles of nucleosome SELEX using random DNA as the initial input library.  In each cycle, I reconstituted nucleosomes from \emph{Xenopus laevis} histone octamer and 147-bp DNA \citep{dyer_reconstitution_2003} methylated at various levels.  This included CpG methylation, methylation at all cytosines, and methylation at half the cytosines.  I extracted the DNA bound to nucleosomes at each cycle for sequencing, then assessed the distribution of mononucleotides and dinucleotides using Fourier transform \citep{lowary_new_1998, zhu_interaction_2018}.  Subsequently, I investigated the effect of histone octamer binding on the enrichment of mononucleotides in the sequencing libraries by producing 9-mer plots.

\section{Results}
\label{sec:emsaselex_results}

% replace with properly-drawn version later
\begin{figure}[htpb]
  \centering
  \includegraphics[width=\textwidth]{selexoverview}
  \caption{Overview of nucleosome EMSA-SELEX procedure}
  \label{fig:selex}
\end{figure}

Four cycles of nucleosome EMSA SELEX (figure~\ref{fig:selex}) were carried out.  147-bp oligonucleotides with a 101-bp central random sequence -- termed lig147 \citep{zhu_interaction_2018} -- constituted the initial input library.  The experiment had four groups: plain (non-methylated) DNA, CpG-methylated DNA, DNA with all cytosines methylated (all-C-methylated DNA), and DNA with half of its cytosines methylated (half-C-methylated DNA).

\subsection{Nucleosome reconstitution}
\label{ssec:reconstnuc}

EMSA in all cycles yielded one strong band between 100 and 200 bp. It also yielded bands corresponding to reconstituted nucleosomes that became fainter as the proportion of histone octamer decreased in the solution (figure~\ref{fig:reconstnuc}).  DNA ligands that were not bound to nucleosomes and ligands that were bound to nucleosomes were eluted and then sequenced.

% add ticks for weights on gels
% put the ratios onto the images instead of in the caption
\begin{figure}[h]
  \centering
  \begin{subfigure}[htpb]{0.4\textwidth}
    \centering
    \includegraphics[width=\textwidth]{reconstnuc_a}
    \caption{Plain DNA}
    \label{fig:reconstnuc_a}
  \end{subfigure}
  \begin{subfigure}[htpb]{0.4\textwidth}
    \centering
    \includegraphics[width=\textwidth]{reconstnuc_b}
    \caption{Half-C-methylated DNA}
    \label{fig:reconstnuc_b}
  \end{subfigure}
  \begin{subfigure}[htpb]{0.4\textwidth}
    \centering
    \includegraphics[width=\textwidth]{reconstnuc_c}
    \caption{All-C-methylated DNA}
    \label{fig:reconstnuc_c}
  \end{subfigure}
  \begin{subfigure}[htpb]{0.4\textwidth}
    \centering
    \includegraphics[width=\textwidth]{reconstnuc_d}
    \caption{CpG-methylated DNA}
    \label{fig:reconstnuc_d}
  \end{subfigure}
  \caption{First-cycle EMSA in 6\% DNA retardation gels for the four experimental groups.  Lanes for all images correspond to these histone:octamer ratios (left to right): (1) 1.48:1, (2) 0.74:1, (3) 0.37:1, (4) 0.19:1, (5) 0.09:1, and (6) DNA without histone octamer added.  These results are representative for all cycles.}
  \label{fig:reconstnuc}
\end{figure}

The frequency of all mononucleotide and dinucleotide sequences at each position along the reads were calculated (figure~\ref{fig:freqplots}).  The data from the fourth cycle of EMSA-SELEX were used in all subsequent analyses because they exhibited the strongest signals.  All DNA libraries bound to nucleosomes showed a clear periodicity in mononucleotide and dinucleotide frequencies, but the periodicity in the plain DNA libraries was less clear.

\begin{figure}[htb]
  \centering
  \begin{subfigure}[htb]{0.45\textwidth}
    \centering
    \includegraphics[width=0.45\textwidth]{emsa_e8_counts}
    \caption{Plain DNA}
    \label{fig:freqplots_e8}
  \end{subfigure}
  \begin{subfigure}[htb]{0.45\textwidth}
    \centering
    \includegraphics[width=0.45\textwidth]{emsa_f8_counts}
    \caption{Half-C-methylated}
    \label{fig:freqplots_f8}
  \end{subfigure}
  \begin{subfigure}[htb]{0.45\textwidth}
    \centering
    \includegraphics[width=0.45\textwidth]{emsa_g8_counts}
    \caption{All-C-methylated}
    \label{fig:freqplots_g8}
  \end{subfigure}
  \begin{subfigure}[htb]{0.45\textwidth}
    \centering
    \includegraphics[width=0.45\textwidth]{emsa_h8_counts}
    \caption{CpG methylated}
    \label{fig:freqplots_h8}
  \end{subfigure}
  \caption{Frequencies of each mononucleotide and dinucleotide along the sequencing reads from libraries corresponding to 1:0.74 octamer:DNA ratios for the fourth cycle of EMSA-SELEX}
  \label{fig:freqplots}
\end{figure}

An additional library of plain DNA bound to nucleosomes prepared with additional rounds of AMPure bead purification to enrich the signals was analysed.  Characterisation of frequencies of mononucleotides at each position along the reads for this library yielded clear periodicities (figure~\ref{fig:enriched_counts}), which were confirmed to be approximately 10 bp by fast Fourier transform (figure~\ref{fig:enriched_power}).  However, this library exhibited biases towards G and T nucleotides.

\begin{figure}[htb]
  \centering
  \begin{subfigure}[htb]{0.8\textwidth}
    \centering
    \includegraphics[width=0.8\textwidth]{enriched-counts}
    \caption{}
    \label{fig:enriched_counts}
  \end{subfigure}
  \begin{subfigure}[htb]{0.8\textwidth}
    \centering
    \includegraphics[width=0.8\textwidth]{enriched-power}
    \caption{}
    \label{fig:enriched_power}
  \end{subfigure}
  \caption{Analysis of a library of plain DNA bound to nucleosomes enriched for signal. (a) shows the nucleotide counts as a function of position along the reads, and (b) shows the fast Fourier transform power spectra for the four nucleotides.}
  \label{fig:enriched}
\end{figure}

In contrast, all DNA libraries not bound to nucleosomes were depleted of this periodic signal.  This periodicity was assessed by fast Fourier transform, and the resulting power spectra showed that the periodicity was near 10 bp (figure~\ref{fig:powerspectra}).  Specifically, half-C-methylated and all-C-methylated DNA showed the strongest signals, % QUESTION: quantification of strength/height of peak?
CpG-methylated DNA showed signals of intermediate strength, and plain DNA showed the weakest signals.  Among the octamer-to-DNA concentrations tested, the 1:0.74 ratio gave rise to the strongest signals, and the signal intensity decreased as the proportion of DNA increased (data not shown).  Libraries corresponding to this octamer:DNA ratio were therefore used for all subsequent analyses.

% kind of a placeholder at the moment...
\begin{figure}[htb]
  \centering
  \includegraphics[width=\textwidth]{emsacycle4all}
  \caption{Power spectra derived from the frequencies of all mononucleotide and dinucleotides.  Libraries are from the fourth cycle of EMSA-SELEX.  The ligand type and the octamer:DNA ratios were varied.}
  \label{fig:powerspectra}
\end{figure}

\subsection{Sequence preferences from fast Fourier transform analysis}
\label{ssec:nuseqpref_fft}
% ORIGINAL PLAN:
%% - The sequence preferences of nucleosome 
%% - Disfavored sequences of nucleosome
%% - Methylation effects on nucleosome’s sequence preference

The phases and the amplitudes resulting from the fast Fourier transforms were then characterised, based on a frequency of 10.2 bp (figure~\ref{fig:radial_mc}).

When the cytosines were methylated, the amplitude for dinucleotides beginning with C increased, and their phases became more aligned.  Specifically, in the plain DNA library, the CA, CC, CG, and CT sequences were spread out over phases of \ang{80} to \ang{260} (figure~\ref{fig:radial_mc_plain}).  In the half-C-methylated and all-C-methylated libraries for the same octamer:DNA ratio, these sequences were spead out over phases of \ang{70} to \ang{140} (figures~\ref{fig:radial_mc_halfmc} and~\ref{fig:radial_mc_allmc}).  These dinucleotides became more aligned to TC.  The phases of the reverse complements of these dinucleotides -- namely GA, GC, and GG -- also became more aligned when the cytosines are methylated.   Using the AA sequence as a reference, the amplitudes for CT and CA were greater in the all-C-methylated library than in the half-C-methylated library, but the amplitude for CG slightly reduced.
% QUESTION: is there a better way to describe the `spread' of the CX sequences to replace eyeballing and saying `yeah, it looks like they're less spread alright'.

These effects were less pronounced for CpG methylation (figure~\ref{fig:radial_mc_cpg}). % topic sentence
Using the AA sequence as a reference, the amplitudes of CC and CT increased.  Similar to the other experimental groups, the amplitude of CG exhibited a slight decrease.  However, the CA, CC, CG, and CT sequences were still spread over a wide range of phases, covering \ang{80} to \ang{160}.

\begin{figure}[htb]
  \centering
  \begin{subfigure}[htb]{0.45\textwidth}
    \centering
    \includegraphics[width=0.45\textwidth]{emsa_e8_radial}
    \caption{Plain DNA}
    \label{fig:radial_mc_plain}
  \end{subfigure}
  \begin{subfigure}[htb]{0.45\textwidth}
    \centering
    \includegraphics[width=0.45\textwidth]{emsa_f8_radial}
    \caption{Half-C-methylated}
    \label{fig:radial_mc_halfmc}
  \end{subfigure}
  \begin{subfigure}[htb]{0.45\textwidth}
    \centering
    \includegraphics[width=0.45\textwidth]{emsa_g8_radial}
    \caption{All-C-methylated}
    \label{fig:radial_mc_allmc}
  \end{subfigure}
  \begin{subfigure}[htb]{0.45\textwidth}
    \centering
    \includegraphics[width=0.45\textwidth]{emsa_h8_radial}
    \caption{CpG methylated}
    \label{fig:radial_mc_cpg}
  \end{subfigure}
  \caption{Effects of cytosine methylation: radial plots showing amplitudes and phases for each dinucleotide}
  \label{fig:radial_mc}
\end{figure}

Certain dinucleotides containing cytosines were enriched when the cytosines were methylated, especially in CpG methylation (figure~\ref{fig:c2nt}).  In particular CC, CG, and GC were enriched in the CpG methylated group.  In constrast, CC was depleted in the all-C-methylated group.  However, in all samples, deviations from a 1:1 ratio were small.

% Use R to replace this with a dot plot later
\begin{figure}[htb]
  \centering
  \includegraphics[width=0.4\textwidth]{c2nt}
  \caption{Enrichment of dinucleotides in the 0.74:1 octamer:DNA ratio libraries for each experimental group.  The number of occurences of each dinucleotide in each library was calculated.  Dividing the count from the bound DNA library by the count from the unbound DNA library gave the ratios shown.} 
  \label{fig:c2nt}
\end{figure}

\subsection{Sequence preferences from sequence \emph{k}-mer counts}
\label{ssec:nuseqpref_kmer}

For each library, the frequency of each possible continuous 9-mer of nucleotides was counted.  With the half-C-methylated, all-C-methylated, and CpG-methylated experimental groups (figure~\ref{fig:kmer_bound}), the unbound libraries were enriched in A and T compared to the bound libraries.  This effect was less significant when all cytosines were methylated (figure~\ref{fig:kmer_bound_all}).  However, both the bound and unbound all-C-methylated libraries were enriched in A and T as compared to the corresponding plain DNA libraries (figure~\ref{fig:kmer_bias}). 

\begin{figure}[htb]
  \centering
  \begin{subfigure}[htb]{0.6\textwidth}
    \centering
    \includegraphics[width=0.6\textwidth]{kmer_CpGubXCpGb}
    \caption{CpG methylated}
    \label{fig:kmer_bound_cpg}
  \end{subfigure}
  \begin{subfigure}[htb]{0.6\textwidth}
    \centering
    \includegraphics[width=0.6\textwidth]{kmer_halfubXhalfb}
    \caption{Half-C-methylated}
    \label{fig:kmer_bound_half}
  \end{subfigure}
  \begin{subfigure}[htb]{0.6\textwidth}
    \centering
    \includegraphics[width=0.6\textwidth]{kmer_allubXallb}
    \caption{All-C-methylated}
    \label{fig:kmer_bound_all}
  \end{subfigure}
  \caption{9-mer count scatter plots showing the effect of nucleosome reconstitution on each experimental group.  Horizontal axes correspond to DNA bound in nucleosomes, while vertical axes correspond to unbound DNA.  Green indicates 9-mers consisting of predominantly adenines, blue cytosines, yellow guanines, and red thymines.  Data for plain DNA not shown as signal was too weak.}
  \label{fig:kmer_bound}
\end{figure}

\begin{figure}[htb]
  \centering
  \begin{subfigure}[htb]{0.6\textwidth}
    \centering
    \includegraphics[width=0.6\textwidth]{kmer_plainbXallb}
    \caption{Bound DNA}
    \label{fig:kmer_bias_bound}
  \end{subfigure}
  \begin{subfigure}[htb]{0.6\textwidth}
    \centering
    \includegraphics[width=0.6\textwidth]{kmer_plainubXallub}
    \caption{Unbound DNA}
    \label{fig:kmer_bias_unbound}
  \end{subfigure}
  \caption{9-mer count scatter plots showing sequence biases.  Horizontal axes correspond to the all-C-methylated libraries, while vertical axes correspond to the plain DNA libraries.  Green indicates 9-mers consisting of predominantly adenines, blue cytosines, yellow guanines, and red thymines.}
  \label{fig:kmer_bias}
\end{figure}

The frequency of continuous stretches of adenines (A-stretches) were also counted in each library (figure~\ref{fig:astretch}).  Methylation was associated with A-stretches being less represented in the libraries containing DNA bound to nucleosomes.  Compared to plain DNA, the half-C-methylated and all-C-methylated groups exhibited greater depletions in A-stretches.  This effect was more pronounced in the CpG methylated group.  Almost all of the groups exhibited ratios below 1.0.

% make this look more glamorous later by changing around the R code.  it's already informative
\begin{figure}[htb]
  \centering
  \includegraphics[width=0.7\textwidth]{astretch}
  \caption{Enrichment of continuous stretches of adenines for each experimental group.  The number of occurences of each k-mer of adenine stretches in each library was calculated (e.g. `5' corresponds to `AAAAA').  Dividing the count from the bound DNA library by the count from the unbound DNA library gave the ratios shown.}
  \label{fig:astretch}
\end{figure}

\chapter{PCR Bias}
\label{ch:pcrbias}

\section{Introduction}
\label{sec:pcrbias_intro}

PCR is a known source of bias in massively parallel sequencing \citep{olova_comparison_2018}.  Among the processes in Illumina sequencing, PCR amplification during library preparation has been identified as the major source of bias.
% The corresponding figure from Aird et al. would be useful here
\citet{aird_analyzing_2011} measured frequencies of amplicons of varying GC content after each step in library preparation for Illumina sequencing, and found that DNA shearing, adapter ligation, gel size selection did not significantly introduce biases in response to GC content.  However, PCR depleted loci with GC content exceeding 65\% to the order of 1\% of the reference loci with intermediate GC content, and it also depleted loci with GC content lower than 12\% to the order of 10\% of the reference loci.  This pattern was observed even for 10 PCR cycles.

There are methods to mitigate this bias.  The choice of DNA polymerase enzyme has an effect.  Phusion DNA polymerase is commonly used because it has higher processivity and fidelity than most polymerases \citep{quail_optimal_2012}. Trying to reduce PCR bias by adding betaine and prolonging the denaturation step leads to a loss of high-AT sequences.  \citet{aird_analyzing_2011} also showed that replacing Phusion with AccuPrime Taq HiFi polymerase improved PCR bias.  Extension at \SI{60}{\celsius} using this enzyme retained more low-GC sequences, but also led to a decrease in yield of GC-rich reads.  \citet{quail_optimal_2012} showed that replacing Phusion with Kapa HiFi led to a more uniformm genome coverage.  However, this also led to a higher error rate, especially in TA-rich regions that Phusion often fails to amplify.

Although it is well-characterised that PCR introduces sequence biases, it is unknown which process within procedures that employ PCR amplification contributes the most to this bias, and to which degree.  Specifically, the biases introduced from the number of PCR cycles and from purification have not been characterised.  Such characterisation would be beneficial in developing a model to quantify PCR bias.  With such a model, these PCR biases can be removed in genomic studies to reveal true signals introduced by the factors tested in the relevant experiments.  PCR bias is especially relevant for nucleosome EMSA-SELEX employed in this project.  Sequence affinities for the nucleosome is weak, making the signal enrichment in the sequencing library weak, and makes it difficult to separate true signals introduced by nucleosome binding from biases introduced by PCR.  Both sequences bound to nucleosomes and sequences that are not exhibit the same PCR-introduced bias, rather than have biases that separate these two groups.

Here I assesed the sequence biases that could be introduced to the initial input library for nucleosome EMSA-SELEX.  I tested sequence biases introduced by the choice of DNA polymerase, choice of reagents, cycles of PCR, and purification of DNA after PCR.  From the resulting sequencing data, I assessed the distribution of k-mer nucleotides.

\section{Results}
\label{sec:pcrbias_results}

The initial input library lig147 was amplified and processed, with conditions varied to test biases introduced by polymerase enzymes, DNA purification, and reagent vendors.  After the PCR products were sequenced, the frequencies of each possible continuous 9-mer of nucleotides was counted.

% ORIGINAL
%% - Different oligonucleotide bias of different PCR enzymes
%% - The variance caused by bottleneck effect 
%% - Dependency of bias on PCR template concentration
%% - Bias caused by reagents from different company
%% - Bias from PCR Purification

\subsection{Biases from enzymes}
\label{ssec:pcrbias_result_enz}

Libraries amplified by Phusion, Phire, Dreamtaq, and Q5 DNA polymerase enzymes were analysed.  For each enzyme, the library from 24 cycles of PCR was compared to the library from 8 cycles of PCR (figure~\ref{fig:kmer_enz}).  All enzymes exhibited biases towards A and T, with Phusion exhibiting the weakest bias and Phire exhibiting the strongest bias.

\begin{figure}[h]
  \centering
  \begin{subfigure}[h]{0.45\textwidth}
    \centering
    \includegraphics[width=0.45\textwidth]{kmer_dreamtaq}
    \caption{DreamTaq}
    \label{fig:kmer_enz_dreamtaq}
  \end{subfigure}
  \begin{subfigure}[h]{0.45\textwidth}
    \centering
    \includegraphics[width=0.45\textwidth]{kmer_phire}
    \caption{Phire}
    \label{fig:kmer_enz_phire}
  \end{subfigure}
  \begin{subfigure}[h]{0.45\textwidth}
    \centering
    \includegraphics[width=0.45\textwidth]{kmer_q5}
    \caption{Q5}
    \label{fig:kmer_enz_q5}
  \end{subfigure}
  \begin{subfigure}[h]{0.45\textwidth}
    \centering
    \includegraphics[width=0.45\textwidth]{kmer_phusion}
    \caption{Phusion}
    \label{fig:kmer_enz_phusion}
  \end{subfigure}
  \caption{9-mer count scatter plots showing the effect of polymerase enzyme.  Horizontal axes correspond to libraries after 24 cycles of PCR, while vertical axes correspond to libraries after 8 cycles of PCR.  Green indicates 9-mers consisting of predominantly adenines, blue cytosines, yellow guanines, and red thymines.}
  \label{fig:kmer_enz}
\end{figure}

\subsection{Biases from purification}
\label{ssec:pcrbias_result_pur}

Eight rounds of purification of the lig147 input library by AMPure beads did not introduce measurable nucleotide biases (figure~\ref{fig:kmer_pur}).

\begin{figure}[h]
  \centering
  \includegraphics[width=0.2\textwidth]{kmer_ampure}
  \caption{9-mer count scatter plot showing the effect of AMPure bead purification.  The horizontal axis corresponds to the library after eight rounds of purification, while the vertical axis corresponds to the library after one round of purification.  Green indicates 9-mers consisting of predominantly adenines, blue cytosines, yellow guanines, and red thymines.}
  \label{fig:kmer_pur}
\end{figure}

\subsection{Biases from reagent vendors}
\label{ssec:pcrbias_result_reagent}

Different combinations of vendors for dNTPs and Phusion High-Fidelity DNA polymerases for PCR were investigated.  The fractions of each nucleotide after 24 rounds of amplification were calculated for each combination (figure~\ref{fig:linearmodel_nt}).

% I'm seriously considering changing these plots to dot plots and the vertical axes to be uniform around 0.20 to 0.30.  I have the data, not the code...
\begin{figure}[h]
  \centering
  \begin{subfigure}[h]{0.45\textwidth}
    \centering
    \includegraphics[width=0.45\textwidth]{linearmodel_a}
    \caption{adenine (A)}
    \label{fig:linearmodel_a}
  \end{subfigure}
  \begin{subfigure}[h]{0.45\textwidth}
    \centering
    \includegraphics[width=0.45\textwidth]{linearmodel_c}
    \caption{cytosine (C)}
    \label{fig:linearmodel_c}
  \end{subfigure}
  \begin{subfigure}[h]{0.45\textwidth}
    \centering
    \includegraphics[width=0.45\textwidth]{linearmodel_g}
    \caption{guanine (G)}
    \label{fig:linearmodel_g}
  \end{subfigure}
  \begin{subfigure}[h]{0.45\textwidth}
    \centering
    \includegraphics[width=0.45\textwidth]{linearmodel_t}
    \caption{thymine (T)}
    \label{fig:linearmodel_t}
  \end{subfigure}
  \caption{Fraction of each nucleotide after 24 cycles of amplification by each combination of dNTP and Phusion High-Fidelity DNA polymerase vendors.}
  \label{fig:linearmodel_nt}
\end{figure}

These fractions were then used to construct a linear model (equation~\ref{eqn:linearmodel}) to describe the effects of dNTP and Phusion High-Fidelity DNA polymerase vendor on the fraction of the nucleotides.

\begin{equation}
  \label{eqn:linearmodel}
  \textrm{fraction of nucleotide} \sim\ a (\textrm{enzyme vendor}) + b (\textrm{dNTP vendor}) + c
\end{equation}

\begin{table}[h]
  \centering
  \begin{tabular}{r}
     \\
    Phusion NEB\\
    dNTP Kapa\\
    dNTP Agilent\\
    dNTP ZymoMeC\\
  \end{tabular}%
  \pgfplotstabletypeset[
  col sep = comma,
  ]{linearmodel.csv}  
  \caption{Coefficients for the linear model to describe the contributions of dNTP and Phusion High-Fidelity DNA polymerase vendors to nucleotide fractions}
  \label{tab:linearmodel_coeffs}
\end{table}

The model was then verified with experimental data (figure~\ref{fig:linearmodel_ver}).

% analyse the data and add how well the correlation is
\begin{figure}[h]
  \centering
  \includegraphics[width=0.4\textwidth]{linearmodel_plot}
  \caption{Verification of linear model. R = 0.9980243}
  \label{fig:linearmodel_ver}
\end{figure}

%\printbibliography

\end{document}
